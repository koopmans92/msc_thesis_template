%%%%%%%%%%%%%%%%%%%%%%%%%%%%%%%%%%%%%%%%%%%%%%%%%%%%%%%%%%%%%%%%%%%
%
% Dissertationen mit LaTeX auf dem edoc-Server
%
% Humboldt-Universitaet zu Berlin
% Computer- und Medienservice
% Arbeitsgruppe Elektronisches Publizieren
% Bezug der Vorlage und der Richtlinien:
%     http://edoc.hu-berlin.de/e_autoren/latex/
%
% Kontakt:
%     E-Mail:
%                   edoc-latex@rz.hu-berlin.de
%     Telefon:              siehe
%     http://edoc.hu-berlin.de/e_autoren/latex/kontakt.php  %
%
%%%%%%%%%%%%%%%%%%%%%%%%%%%%%%%%%%%%%%%%%%%%%%%%%%%%%%%%%%%%%%%%%%%%%%%
%
% Das folgende Template muss fuer die Publikation von digitalen       %
% Dissertationen in LaTeX an der Humboldt-Universitaet benutzt werden.%
%
% Aendern Sie den Name dieser Datei auf ihren_nachname.tex.           %
%
% Die mit einem Stern (*) gekennzeichnete Teile sind optional;        %
% falls Sie sie nicht verwenden moechten, sind entsprechende Zeilen   %
% zu entfernen.
%
%%%%%%%%%%%%%%%%%%%%%%%%%%%%%%%%%%%%%%%%%%%%%%%%%%%%%%%%%%%%%%%%%%%%%%%

% \listfiles    % Erstellt eine Liste von allen benutzten Dateien
% zusammen mit ihrer Versionen
% und ggf. einer kurzen Beschreibung
% Sie wird in die log-Datei geschrieben

\documentclass[12pt,a4paper,twoside%,openright% die Verwendung von DIN-A4-Format ist pflicht!
]{scrreprt}
\usepackage[top=2cm, bottom=2.5cm, left=3cm, right=3cm,footskip=1.25cm]{geometry}

%notwendige Pakete
\usepackage[ngerman, english]{babel}    % mehrsprachiger Textsatz
% babel: letzte Sprache in Optionen zeigt die Sprache des Dokumentes
% und kann durch den Befehl \selectlanguage{} geaendert werden
% Passen Sie die Optionen des babel-Paketes nach Bedarf an!
\usepackage[utf8]{inputenc}       % Eingabekodierung Parameter latin1 darf geaendert werden
\usepackage[T1]{fontenc}                % Schriftenkodierung
\usepackage{graphicx}                       % zum Einbinden von Grafiken
\usepackage{lmodern}                        % Ersatz fuer Computer Modern-Schriften
\usepackage{marvosym}                       % zum besseren Aussehen am Bildschirm
\usepackage{subcaption}
\usepackage{amssymb}
\usepackage{lineno}
\usepackage{slashed}
\usepackage{amsmath}
\usepackage{empheq}
%\usepackage{fancyhdr}
\usepackage{scrlayer-scrpage}
\usepackage{overpic}
\usepackage{tikz}
\usetikzlibrary{fit}
\usepackage{sidecap}
\usepackage{siunitx}
\usepackage{booktabs}
%\usepackage[nottoc]{tocbibind}
%\usepackage[force]{feynmp-auto}
%\usepackage{feynmf}
%\usepackage{feynmp}
\usepackage[babel,german=quotes]{csquotes}
\usepackage{textcomp}
\usepackage{breqn}
\usepackage{bm}
\usepackage{scrhack}
\setcapindent{10pt}
%\usepackage{xcolor}
%\usepackage[format=plain]{caption}

% biblatex and modifications
\usepackage[hyperref=true, backend=biber, parentracker=true, backref=true, sorting=none, citestyle=numeric-comp]{biblatex}
\ExecuteBibliographyOptions{isbn=false, doi=true, url=false}
\ExecuteBibliographyOptions{backref=true}
\ExecuteBibliographyOptions{abbreviate=true}
\ExecuteBibliographyOptions{eprint=false}
%\ExecuteBibliographyOptions{giveninits=true}
\addbibresource{bibliography.bib}

\DefineBibliographyStrings{english}{%
   bibliography = {References},
}
%\usepackage{natbib}
%\usepackage[clearempty]{titlesec}\
\PassOptionsToPackage{hyphens}{url}
%-Eingabe der Metadaten des Titelblattes--------------------------

%-Daten des Autors / Authors Data---------------------------------

\newcommand{\dcauthorpre}{~} 
\newcommand{\dcauthorsurname}{Vorname}
\newcommand{\dcauthorname}{Name} 
\newcommand{\dcauthoradd}{geboren am 12.12.1212 in Berlin}

%-Titel und Untertitel / Title and subtitle-----------------------

\newcommand{\dctitle}{TITLE} 
%\newcommand{\dctitle}{Double Slit Interferometer for Transverse Beam Size Measurements at BESSY II} 
\newcommand{\dcsubtitle}{~}  
% Falls dcsubtitle NICHT verwendet werden soll, {\dcsubtitle}{~} eingeben.

%-Eingabe der Betreuuernahmen / Names of the consultants---------

\newcommand{\dcconsulta}{Prof.\ Dr.\ ...} 
\newcommand{\dcconsultb}{Dr.\ ...} 
\newcommand{\dcconsultc}{Dr.\ ...} 

%-Eingabe der Gutachternamen / Names of the approvals-------------

\newcommand{\dcapprovala}{Prof.\ Dr.\ Andreas Jankowiak} 
\newcommand{\dcapprovalb}{Prof.\ Dr.\  ...} 
\newcommand{\dcapprovalc}{Prof.} 

%-Information zur Universitaet------------------------------------

\newcommand{\dcdegree}{Master of Science (M.\,Sc.)} 
\newcommand{\dcsubject}{Physik} 
\newcommand{\dcfaculty}{Mathematisch-Naturwissenschaftlichen Fakult\"at}
\newcommand{\dcinstitute}{Institut f\"ur Physik}
\newcommand{\dcuniversity}{Humboldt-Universit\"at zu Berlin}% \\ Hemlholtz Zentrum Berlin}
\newcommand{\dcdean}{}
\newcommand{\dcpresident}{}

%-Pruefungsdaten: eingereicht und mdl. Pruefung-------------------
%-data of submission and oral exam--------------------------------

\newcommand{\dcdatesubmitted}{10. Februar 1999} %auch wenn nicht auf dem Titelblatt, bitte erf�llen!
\newcommand{\dcdateexam}{2. Juli 1999} 

%-deutsche Schlagwoerter / german keywords------------------------

\newcommand{\dckeydea}{Schlagwort 1}
\newcommand{\dckeydeb}{Schlagwort 2}
\newcommand{\dckeydec}{Schlagwort 3}
\newcommand{\dckeyded}{Schlagwort 4}

% Folgende Zeile bitte nicht aendern!
\newcommand{\dckeywordsde}{\vfill \raggedright {\textbf{Schlagw\"orter:}}\\ \dckeydea, \dckeydeb, \dckeydec, \dckeyded \\}

%-englische Schlagwoerter / english keywords----------------------

\newcommand{\dckeyena}{keyword 1}
\newcommand{\dckeyenb}{keyword 2}
\newcommand{\dckeyenc}{keyword 3}
\newcommand{\dckeyend}{keyword 4}

% Folgende Zeile bitte nicht aendern!
\newcommand{\dckeywordsen}{\vfill \raggedright {\textbf{Keywords:}}\\ \dckeyena, \dckeyenb, \dckeyenc, \dckeyend \\}

\newcommand{\dcpdfsubject}{Dissertation}                          % Bitte ALLE Angaben erfuellen!
\usepackage{ifpdf}

\ifpdf
% das kann man benutzen, wenn man andere Formate benutzen will
% \DeclareGraphicsExtensions{{.pdf}}   %Endung der Grafiken, wenn nicht pdf
% die folgenden Angaben sind im PDF unter Datei | Dokumenteigenschaften 
% in Acrobat / Acrobat Reader sichtbar
% Aendern Sie bitte die Daten, wo noetig!
\usepackage[%
	pdftitle={\dctitle},
	pdfauthor={\dcauthorsurname\ \dcauthorname},
	pdfsubject={\dcpdfsubject}, % optional
	pdfkeywords={\dckeydea, \dckeydeb, \dckeydec, \dckeyded},
	pdfpagemode=UseOutlines,
	unicode=true,
	pdfstartview={FitH},
	colorlinks=true,					% bitte nicht �ndern!
	linkcolor=black,					% bitte nicht �ndern!
	filecolor=black,					% bitte nicht �ndern!
	urlcolor=black,						% bitte nicht �ndern!
	citecolor=black,					% bitte nicht �ndern!
	%pdftex=true,              % bitte nicht �ndern!
	plainpages=false,         % bitte nicht �ndern!
	hypertexnames=true,      % bitte nicht �ndern!
	pdfpagelabels=true,       % bitte nicht �ndern!
	hyperindex=true]{hyperref}% bitte nicht �ndern!
\else
  % hier kann mann eventuelle Befehle umdefinieren
  % die nur f�r pdfLaTeX vorgesehen sind
  % und das richtige Kompilieren durch den normalen LaTeX verhindern
	\newcommand{\texorpdfstring}[1]{#1}
\fi                     % hyperref!!!!

% Stuff after(!) hyperref
%\DeclareFieldFormat*{title}{\usebibmacro{string}{\mkbibquote{#1}}}
\newbibmacro{string+doiurlisbn}[1]{%
  \iffieldundef{url}{%
    \iffieldundef{doi}{%    
      #1
     }{%
       \href{http://dx.doi.org/\thefield{doi}}{#1}%
    }%
  }{%
      \href{\thefield{url}}{#1}%
  }%
}
\DeclareFieldFormat*{title}{\usebibmacro{string+doiurlisbn}{\mkbibquote{#1}}}
\renewcommand\newunitpunct{\addcomma\addspace} % komma statt punkt!!!!

% cleverref for easy referencing
\usepackage[capitalize]{cleveref}
\Crefname{figure}{Figure}{Figures}
\crefname{figure}{Fig.}{Figs.}
\Crefname{table}{Table}{Tables}
\crefname{table}{Tab.}{Tabs.}
% remove parenthesis from equation references
\creflabelformat{equation}{#2#1#3}

\clubpenalty10000
\widowpenalty10000
\displaywidowpenalty=1000000

% Supress warning: multiple pdfs with page group included ....
\pdfsuppresswarningpagegroup=1

% header: ro chapter, lo section
\automark[]{chapter}
\automark*[section]{}

% nicer primed vector
\newcommand{\pvec}[1]{\vec{#1}\mkern2mu\vphantom{#1}}
\emergencystretch=1em
%-Eigene Trennregeln*---------------------------------------------

% % Tragen Sie Ihre eigenen Trennregel ein:
\hyphenation{Bei-spiel, �-ko-lo-gie}

%-zusaetzliche Kommandos*-----------------------------------------

%\include{command}



%-Dokument--------------------------------------------------------

\begin{document}
\unitlength = 1mm
% Es muss zitiert werden koennen! Im Vorspann roemisch,
% Im Hauptteil benutzt man arabische Nummerierung.
\pagenumbering{roman}

%-Titelblatt------------------------------------------------------
\selectlanguage{english}
%\begin{titlepage}
%----------Generierung der Titelseite-----bitte nicht veraendern!--------------------
\author{ }%%von \\ \dcauthorpre\ \dcauthorname\ \dcauthorsurname\ \\ \dcauthoradd}

%----------
\title{ \vspace{-1cm}
%\vspace{0.5cm}
\dctitle \\
\vspace{0.5cm}
%\large{\dcsubtitle} \\
\vspace{0.5cm} 
\Large{MASTER THESIS}\\
	\vspace{1cm} 
%%\vspace{0.5cm}
\Large{in fulfillment of the requirements for the degree \\
%\dcdegree\\ im Fach \dcsubject \\\vspace{0.5cm}
\dcdegree~in Physics \\ \vspace{0.5cm}}
\includegraphics[width=5cm]{husiegel}\\
\vspace{0.5cm} %%eingereicht an der \\
 \large{ 
\dcuniversity \\
Faculty of Mathematics and Natural Sciences \\
Department of Physics\\
%\dcuniversity \\
}}
%-----------------
\date{\vspace{-0cm}
%\raggedright{
%Pr\"asident der Humboldt-Universit\"at zu Berlin:\\
%\dcpresident \vspace{-0.3cm}
%}\vspace{0.5cm}\\
%
%\raggedright{
%Dekan der \dcfaculty:\\
%\dcdean \vspace{-0.3cm}
%}\vspace{0.5cm}\\
%
% auskommentiert weil nicht standard
%\raggedright{
%Betreuer:
%\begin{enumerate} 
%\item{\it\dcapprovala} \vspace{-0.3cm}
%\item{\it\dcapprovalb} \vspace{-0.3cm}
%\item{\it\dcapprovalc} \vspace{-0.3cm}
%\end{enumerate}} \vspace{0.5cm}
% \raggedright{
%% Gutachter:
%% \begin{enumerate} 
%% \item{\dcapprovala} \vspace{-0.2cm}
%% \item{\dcapprovalb} \vspace{-0.2cm}
%% \end{enumerate}} \vspace{0.5cm}
%-----------------
\raggedright{\normalsize{
\begin{tabular}{lll}
submitted by: & & \dcauthorsurname\ \dcauthorname\\
born:	 & & Birthdate\\
		 & & \\
reviewer:	 & & \dcapprovala\\
		 & & \dcapprovalb\\
		 & & \\
submitted:  & & $30^{\mathrm{th}}$ November 2017\\ % wenn nicht in der Pruefungsordnung, die Zeile bitte auskommentieren %DATUM anpassen!!!
%%% if neccessary:
%& & \\
%corrected version: & & $12^{\mathrm{th}}$ February 2018 \\
%%Tag der m\"undlichen Pr\"ufung: & & \dcdateexam
\end{tabular}}}
%\\
}%%\vspace{-2.5cm}
%-------------------------------------                         % Bitte KEINE Aenderungen vornehmen!
\maketitle
%\thispagestyle{empty}~
%\end{titlepage}

%\cleardoublepage
%\newpage
%\thispagestyle{empty}~


%-Zusammenfassung / Abstract*-------------------------------------

%-englische-Zusammenfassung---------------------------------------

\selectlanguage{english}

\begin{abstract}
\setcounter{page}{3} % Nach Bedarf anpassen!
english abstract
% hier werden die englische Schlagw�rter aus Metadaten �bernommen
%\dckeywordsen				
\end{abstract}
\newpage
\thispagestyle{empty}~

%-deutsche Zusammenfassung----------------------------------------

\selectlanguage{ngerman}

\begin{abstract}
\setcounter{page}{5} % Nach Bedarf anpassen!
 deutscher Abstract
% hier werden die deutsche Schlagw�rter aus Metadaten �bernommen
%\dckeywordsde
\end{abstract}
\newpage
\thispagestyle{empty}~

% upgrade  <-> aufruestung, verbesserung ?!
% Bunch    <-> paket, buendel ?!
% Pinhole  <-> nadelloch; 
% pinhole camera <-> Lochkamera ?!
\selectlanguage{english}               % Bitte an die Sprache denken!!!
%\setcounter{page}{4}                   %   Bitte an die Seitenzahl denken!!!

%-Widmung*--------------------------------------------------------

% \chapter*{Widmung}
Hier folgt dann eine Widmung.

%-Inhaltsverzeichnis----------------------------------------------

\selectlanguage{english}
\cleardoublepage
\tableofcontents
\clearpage


\KOMAoptions{open=right}
%\cleardoublepage

%-Hauptteil-------------------------------------------------------

\pagenumbering{arabic}
\pagestyle{scrheadings}                  % bzw. ist fancyhdr zu benutzten

%-Kapitel---------------------------------------------------------

% part ist optional, bitte ggf. loeschen
% \part{Teil1}
%\linenumbers
%\cleardoublepage

\include{introduction}
%\KOMAoptions{open=any}
\chapter{Chapter 1 \label{ch:ch1}} 




%\include{chapter02}
\chapter{Conclusion \label{chap:summary}} 


%-Anhang----------------------------------------------------------

\appendix

%-Kapitel des Anhangs---------------------------------------------

\chapter{Appendix 1 \label{ch:app1}}




%\include{appendixB}
%usw.


%-Literaturverzeichnis--------------------------------------------

%\nocite{*}
%die Verwendung von bibtex ist Pflicht!!!
\addcontentsline{toc}{chapter}{\listfigurename}
\listoffigures
\clearpage

\addcontentsline{toc}{chapter}{\listtablename}
\listoftables
\clearpage

\printbibliography[heading=bibintoc]
\clearpage
%\bibliography{bibliography}
%\bibliographystyle{unsrtdinmod}        %bzw. unsrtdin, alphadin, abbrvdin, plaindin, elsarticle-num


%-Abkuerzungen*---------------------------------------------------
 
%\chapter*{Abk�rzungen}


\begin{tabular}{ll}
 Abk�rzung & Erkl�rung    \\
 \hline
      z.B. & zum Beispiel \\
\end{tabular}



%-Danksagung*-----------------------------------------------------

%%-Danksagung------------------------------------------------------

\chapter*{Danksagung}
Hier folgt dann eine Danksagung.


%-Lebenslauf*-----------------------------------------------------

%\chapter*{Lebenslauf}

%

\begin{tabular}{ll}

Name: & \dcauthorname  \dcauthorsurname \\

10.1994--09.1995 & Studium an der Humboldt-Universit"at zu Berlin \\

 & in der Fachrichtung Biologie\\

10.1995--11.1996 & Wissenschaftliche Mitarbeiterin an \\

 & der Humboldt-Universit"at zu Berlin,\\

 & Lehrstuhl Prof. XY, \\

 & Institut f"ur Biologie\\

\end{tabular}


%-Selbstaendigkeiterklaerung--------------------------------------

\chapter*{Statement of Authorship}

\selectlanguage{english}

%Text der Selbst\"andigkeitserkl\"arung.


I hereby declare that I have authored the present master thesis independently, and that I have not used any sources and means other than those specified.

% deutsche vorlage
%Hiermit versichere ich, dass ich die vorliegende Arbeit selbst\"andig verfasst und keine anderen als die angegebenen Quellen und Hilfsmittel verwendet habe.

% Felix MSc:
%I hereby confirm that I have authored this master's thesis independently and without use of others than the indicated sources. All passages which are literally or in general matter taken out of publications or other sources are marked as such.

\vspace{2\baselineskip}
%\noindent Berlin, $27^{\mathrm{th}}$ August 2017 \hfill \rule{10cm}{1pt}\\
%\ \hfill \dcauthorsurname\ \dcauthorname

%%%%%%%%%%%%%%%%%%
%DATUM anpassen!!!!
\noindent Berlin, $30^{\mathrm{th}}$ November 2017 
%
\vspace{0.1cm}
\\
\hspace*{\fill}\begin{tabular}{@{}l@{}}\hline
	\makebox[5cm]{\dcauthorsurname\ \dcauthorname}
\end{tabular}

%-----------------------------------------------------------------

\end{document}
