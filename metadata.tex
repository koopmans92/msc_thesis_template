%-Eingabe der Metadaten des Titelblattes--------------------------

%-Daten des Autors / Authors Data---------------------------------

\newcommand{\dcauthorpre}{~} 
\newcommand{\dcauthorsurname}{Vorname}
\newcommand{\dcauthorname}{Name} 
\newcommand{\dcauthoradd}{geboren am 12.12.1212 in Berlin}

%-Titel und Untertitel / Title and subtitle-----------------------

\newcommand{\dctitle}{TITLE} 
%\newcommand{\dctitle}{Double Slit Interferometer for Transverse Beam Size Measurements at BESSY II} 
\newcommand{\dcsubtitle}{~}  
% Falls dcsubtitle NICHT verwendet werden soll, {\dcsubtitle}{~} eingeben.

%-Eingabe der Betreuuernahmen / Names of the consultants---------

\newcommand{\dcconsulta}{Prof.\ Dr.\ ...} 
\newcommand{\dcconsultb}{Dr.\ ...} 
\newcommand{\dcconsultc}{Dr.\ ...} 

%-Eingabe der Gutachternamen / Names of the approvals-------------

\newcommand{\dcapprovala}{Prof.\ Dr.\ ...} 
\newcommand{\dcapprovalb}{Prof.\ Dr.\ ...} 
\newcommand{\dcapprovalc}{Prof.} 

%-Information zur Universitaet------------------------------------

\newcommand{\dcdegree}{Master of Science (M.\,Sc.)} 
\newcommand{\dcsubject}{Physik} 
\newcommand{\dcfaculty}{Mathematisch-Naturwissenschaftlichen Fakult\"at}
\newcommand{\dcinstitute}{Institut f\"ur Physik}
\newcommand{\dcuniversity}{Humboldt-Universit\"at zu Berlin}% \\ Hemlholtz Zentrum Berlin}
\newcommand{\dcdean}{}
\newcommand{\dcpresident}{}

%-Pruefungsdaten: eingereicht und mdl. Pruefung-------------------
%-data of submission and oral exam--------------------------------

\newcommand{\dcdatesubmitted}{10. Februar 1999} %auch wenn nicht auf dem Titelblatt, bitte erf�llen!
\newcommand{\dcdateexam}{2. Juli 1999} 

%-deutsche Schlagwoerter / german keywords------------------------

\newcommand{\dckeydea}{Schlagwort 1}
\newcommand{\dckeydeb}{Schlagwort 2}
\newcommand{\dckeydec}{Schlagwort 3}
\newcommand{\dckeyded}{Schlagwort 4}

% Folgende Zeile bitte nicht aendern!
\newcommand{\dckeywordsde}{\vfill \raggedright {\textbf{Schlagw\"orter:}}\\ \dckeydea, \dckeydeb, \dckeydec, \dckeyded \\}

%-englische Schlagwoerter / english keywords----------------------

\newcommand{\dckeyena}{keyword 1}
\newcommand{\dckeyenb}{keyword 2}
\newcommand{\dckeyenc}{keyword 3}
\newcommand{\dckeyend}{keyword 4}

% Folgende Zeile bitte nicht aendern!
\newcommand{\dckeywordsen}{\vfill \raggedright {\textbf{Keywords:}}\\ \dckeyena, \dckeyenb, \dckeyenc, \dckeyend \\}

\newcommand{\dcpdfsubject}{Dissertation}